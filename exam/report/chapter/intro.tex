\chapter*{Introduction}
The main aim is to create a deception component generator for a ldap server. Defensive deception is one of the methods that cybersecurity experts have started to use. This technique consists in creating fake services and components that appear as valuable targets to attackers.
In that way defenders can divert the attacker’s attention and resources away from critical assets.\\\\
This approach is known as \textbf{Cyber denial and deception} and has as main focus to delay the attack operation, understand the tools and techniques used by the attackers and push the threats into a safe area.\\
By the definition given, deception's goals are two-fold:
\begin{itemize}
    \item provide false or misleading information to the attacker, but capable to give him the belief and confidence that the attack is being performed on real data. This is called \emph{Deception}.
    \item create uncertainty about the reality of the environment that the attacker is facing, to slow down the attacking operations or lead it to waste its time and resources. That is \emph{Denial}.
\end{itemize}
\\\\
Thaks to that attackers spend time and effort trying to compromise these fake elements, leaving less capacity to target the actual valuable asset. 
\\\\
The goal here is to create a deception component for a ldap server. A ldap server could be used to access information about an organization, such as who the employees are and what role they have or also password an personal information if present in the directory. 
\\\\
To reach the goal i used docker to have an easy way to deploy the service and at the same time allow personal configuration and create, update and populate the directory managed by the LDAP protocol. 
\\\\
Now the cybersecurity expert can quickly generate the data, create the docker container, run it and have a distraction for a potential attacker. 
\\\\
In the following i'll describe the whole process in detail.


